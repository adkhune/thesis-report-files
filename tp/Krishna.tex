\documentclass[12pt]{report}
%\documentclass[masters, reqno]{csuthesis}
\setcounter{secnumdepth}{1}
\usepackage{hyperref}
\hypersetup{
    colorlinks=true, %set true if you want colored links
    citecolor=black,
    filecolor=black,
    linkcolor=black, %choose some color if you want links to stand out
    urlcolor=black
}

\usepackage[margin=0.5in]{geometry}      % default margins are too big
\usepackage{graphicx}                  % for \includegraphics
\usepackage{listings}                  % for typesetting source code
\lstset{language=Python}
\usepackage{mathtools}                 % for better typesetting of math
%\usepackage[round]{natbib}            % for using different bibliography styles
\bibliographystyle{ieeetr}  
\usepackage{url}
%\usepackage{amsmath}
\usepackage{verbatim}

\usepackage{caption}
\usepackage{subcaption}
\usepackage{listings}
\usepackage{color}

\definecolor{dkgreen}{rgb}{0,0.6,0}
\definecolor{gray}{rgb}{0.5,0.5,0.5}
\definecolor{mauve}{rgb}{0.58,0,0.82}

\lstset{frame=tb,
  language=Java,
  aboveskip=3mm,
  belowskip=3mm,
  showstringspaces=false,
  columns=flexible,
  basicstyle={\small\ttfamily},
  numbers=none,
  numberstyle=\tiny\color{black},
  keywordstyle=\color{black},
  commentstyle=\color{black},
  stringstyle=\color{black},
  breaklines=true,
  breakatwhitespace=true
  tabsize=3
}                 % for typesetting source code

\usepackage{xcolor}
\usepackage{xparse}
\NewDocumentCommand{\framecolorbox}{oommm}
 {% #1 = width (optional)
  % #2 = inner alignment (optional)
  % #3 = frame color
  % #4 = background color
  % #5 = text
  \IfValueTF{#1}
   {\IfValueTF{#2}
    {\fcolorbox{#3}{#4}{\makebox[#1][#2]{#5}}}
    {\fcolorbox{#3}{#4}{\makebox[#1]{#5}}}%
   }
   {\fcolorbox{#3}{#4}{#5}}%
 }
 
%%%%%%%%%%%%%%%%%%%%%%%%%%%%%%%%%%%%%%%%%%%%%%%%%%%%%%%%%%%%%%%%%%%%%%%%%%%%%%%%%%%%%%%%%%%%%%%%%%%%%%%
\begin{document}

\title{Report-2\\
Understanding the Nuances of Energy Savings in Mobiles using Code Offloading on Cloud}

\author{Aditya Khune}

\date{\today}  % Leave this line out to use the current date.
\maketitle
%%%%%%%%%%%%%%%%%%%%%%%%%%%%%%%%%%%%%%%%%%%%%%%%%%%%%%%%%%%%%%%%%%%%%%
\tableofcontents


%%%%%%%%%ABSTRACT%%%%%ABSTRACT%%%%%%%ABSTRACT%%%%%%%%%ABSTRACT%%%%%%%%%%%%%%%%%%%%%%%%%%%%%%%%%%%%%%%%

\begin{abstract}
Computation Offloading or Cyber foraging is a decade-old concept, which is today being widely considered for saving battery power in Compute Intensive Smartphone applications. Researchers say that with the right offloading decision the power consumed by CPU can be saved by offloading much of its processing either partially of fully to a Cloud. Huge amount of work is already published in this area, which shows positive results for offloading with right offloading decisions. But critiques doubt the idea of saving battery power with offloading because of various complexities involved in the process.
We believe that it is important to critically examine each aspect which is involved in overall smartphone battery problem. In this paper we have done application oriented study of Offloading in varying real-time network scenarios (3G, 4G, WiFi). We have presented both positive and negative sides of offloading solution with the help of experimental results obtained with a large set of applications on 3 different Smartphones (Amazon Fire phone, Samsung S3 and LG Nexus). While we do believe that Cloud has great advantages; but when it comes to energy savings our results indicate that Offloading does not provide clear benefits over local processing always and we argue that use of offloading should be restricted to a small set of applications.
Experiments conducted on cellular networks (e.g. 3G, 4G, WiFi) demonstrate that our proposed technique can significantly improve user experience and ensure energy savings in offloading system.
Applications that may benefit from Offloading as suggested by important publications are as follows: natural language translators, speech recognizers, optical character recognizers, image processors, image search online games, video processing and editing, navigation, face recognition, augmented reality, etc. These applications consume large mobile battery, memory, and computational resources.
\end{abstract}

%1%%%%%%INTRODUCTION%%%%%%%%INTRODUCTION%%%%%%%%%%INTRODUCTION%%%%%%%INTRODUCTION%%%%%%%%%%%%%%%%%%%%%%%
\chapter{WHY COMPUTATION OFFLOADING IS A BAD IDEA WHEN OUR AIM IS TO SAVE BATTERY POWER ON SMARTPHONES} %Chapter 1\\
\label{chap:Introduction}
There are many components in smartphone which are responsible for the overall poor battery performance. CPU, GPU, LCD screen, Wi-Fi, GPS, Camera, Various sensors, speakers etc. Out of all these offloading mainly focuses on the CPU processing. In offloading we are using a network component WiFi/3G/4G while trying to compensate on CPU processing energy. With the advances in networking technology we have 4G available to us, but it is seen that 4G consumes more energy than 3G and WiFi.
Network Inconsistency: Most important research work in offloading Decision Engines require a consistent network performance for offloading. However, such consistency is difficult to achieve because of frequent mobile user movements and unstable network quality. the power consumed by the radio interface is known to contribute a considerable fraction of the total device power. (please read: A close examination of performance and power characteristics of 4g lte networks)
With recent advent of 4G LTE networks, there has been increased interest in the offloading domain, but Research shows that LTE is as much as 23 times less power efficient compared to WiFi, and even less power efficient than 3G [huang2012close].

Smartphones: Most modern-age smartphones have powerful processors, up to 1 GB of memory and ample secondary storage, such users are less likely to require frequent mobile cloud support as compared to users with feature phones. 
Application: if the data size is too large and application data is unavailable in the cloud, the mobile side computation is encouraged. coz this scenario involves higher execution time and consumes high energy in terms of communication which may negate the benefits of offloading.

\section{EXPERIMENTS AND RESULTS}
\subsection{Cloud-based Web Browser:}
Cloud-based Web browsers([AmazonSilk, ChromeBeta, OperaMini, wang2013accelerating]) use a split architecture where processing of a Mobile web browser is offloaded to cloud partially; It involves cloud support for most browsing functionalities such as execution of JavaScript (JS), image transcoding and compression, parsing and rendering web pages.
 Research shows that CB does not provide clear benefits over device-based browser (e.g. Local Processing) either in energy or download time. Offloading JS to the cloud is not always beneficial, especially when user interactivity is involved [sivakumar2014cloud].
Already there are a number of cloud-based mobile web browsers that are available in the industry e.g. Amazon Silk [AmazonSilk], Opera Mini [OperaMini], Chrome beta [ChromeBeta] etc.
\begin{figure}[h]
  \centering
  \includegraphics[width=5in]{"GIMP Images/MobileBrowserDataImpact".png}
  \caption{MobileBrowserDataImpact}
  \label{fig:MobileBrowserDataImpact}
\end{figure}

\begin{figure}[h]
  \centering
  \includegraphics[width=5in]{"GIMP Images/MobileBrowserUserInteractivity".png}
  \caption{MobileBrowserUserInteractivity}
  \label{fig:MobileBrowserUserInteractivity}
\end{figure}

\subsection{Cloud-based Mobile Gaming:}
Gaming-on-Demand is an emerging trend of the gaming industry which uses Cloud. Video games are executed in the private cloud servers, and the gaming video frames are transmitted over the Internet to desktop PCs, Smartphones or interactive televisions. Game player’s interactions are sent to the cloud server over the same network. For our experiments we have used an open-source cloud gaming platform called GamingAnywhere [gamingeverywhere] also we have done experimentations with commercial cloud gaming solutions by companies such as Gaikai [Gaikai], G-Cluster [gcluster].
During screen-on periods, the CPU on average, is Idle for 80.0% of time, signifying that user activities do not use a lot of computational power. (2) On average, the total CPU busy time during screen-on and screen-off periods is
10.2%
Our results show that if 60% of battery will be used by the Mobile Screen and say 30% of battery by CPU. By offloading we might lower the percentage of CPU to 20% but then we need extra energy for the network components. So there is no significance amount of battery savings can be achieved by Cloud gaming if any. There are chances offloading may cost higher energy.

\begin{figure}[h]
  \centering
  \includegraphics[width=5in]{"GIMP Images/MobileGamesOffloadedProcessing".png}
  \caption{MobileGamesOffloadedProcessing}
  \label{fig:MobileGamesOffloadedProcessing}
\end{figure}

\begin{figure}[h]
  \centering
  \includegraphics[width=5in]{"GIMP Images/MobileGamesLocalProcessing".png}
  \caption{MobileGamesLocalProcessing}
  \label{fig:MobileGamesLocalProcessing}
\end{figure}

\subsection{Voice Recognition}
Results show that Voice Recognition can be done on local system efficiently. Google translate is one of the app which uses cloud to do the voice recognition. It also has an offline translation mode which 


\chapter{WHERE COMPUTATION OFFLOADING CAN BE USEFUL}
In this section we have shown how the cloud computing can benefit for the applications which may not be compute intensive but data intensive for example Torrents.
Application: if the data size is too large and application data is in the cloud, then cloud side computation is encouraged. coz this scenario involves higher execution time and consumes high energy in terms of communication which may negate the benefits of offloading.
\section{EXPERIMENTS AND RESULTS}
\subsection{Torrents:}
Kelenyi et al. [6] proposed a strategy to save energy of handheld devices by offloading the download processing on the Cloud. In their strategy the cloud servers are used as a BitTorrent client to download torrent pieces on behalf of a mobile handheld device. While the cloud server downloading the torrent pieces, the mobile handheld device switch to sleep mode until the cloud finishes the torrent processes and upload the torrent file in one shot to the handheld device. This strategy saves energy of smartphones because downloading torrent pieces from torrent peers consumes more energy than downloading a one burst of pieces from the cloud. The results obtained by them are as follows

\subsection{Zipper:}
\subsection{Machine Learning (for Face Recognition, Health monitoring applications)}
\subsection{Image Searching on Cloud Data}
\subsection{Matrix Calculator}

\chapter{Conclusion}
To Cloud can be useful when we need to train neural networks for machine learning applications, the data is present already in the cloud. Google Image gives unlimited amount of space for all the users, to handle such massive data definitely can save battery consumption if handles in cloud. Machine Learning applications will be a great success in all aspects including energy for the mobile devices. We have seen that not only the data and complexity of the application, but also users interaction with the application, the usage of screen intensive applications.

\bibliographystyle{plainnat}  % or plain, or many other possibilities
\bibliography{bibnarasimha.bib}


\end{document}